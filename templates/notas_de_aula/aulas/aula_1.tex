
\newpage
\pagestyle{fancy}
\pagenumbering{arabic}
\setcounter{page}{3} % Define a página inicial.

\section{Aula 1 - Título da aula}
\hspace{0.5cm}Aqui você vai incluir as suas anotações. Sugiro usar o VSCode para editar se você for um marinheiro de primeira viagem.

as suas anotações estarão aqui, etc. etc.

o template conta com ambientes de definição, teoremas, observações.... Sugiro conferir no config.tex na pasta do preâmbulo. Mas seguem alguns exemplos

\begin{definicao}{Deinição de x}{def:defx}
	Define-se x como tal tal tal
\end{definicao}

\begin{observacao}{sobre x}{obs:rotulo}
	perceba que da definição, concluímos isso e aquilo. Veja só
	\begin{exemplo}{Isso e aquilo}{ex:rotulo2}
		isso e aquilo menos aquilo outro
	\end{exemplo}
\end{observacao}

\hl{\textbf{Sugiro criar, na pasta aulas, arquivos .tex organizando suas aulas. Por exemplo aula\_2.tex, aula\_3.tex, $ \dots $, aula\_n.tex ($ n \in \mathbb{N} $)}}.

Ao final da matéria, as notas de aula podem ser reestruturadas em capítulos ou algo do tipo. Vai do gosto do cliente!