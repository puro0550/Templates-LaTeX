%======================================
% CONFIGURAÇÕES DO DOCUMENTO
%======================================

%====== PACOTE PARA CAIXAS DE TEOREMAS ======%
\usepackage[most]{tcolorbox}
\tcbuselibrary{theorems}

%====== ESTILO PADRÃO PARA TEOREMAS ======%
\tcbset{
  mytheostyle/.style={
    enhanced,
    breakable,
    colback=white,       % fundo branco
    colframe=black,      % borda preta
    fonttitle=\bfseries, % título em negrito
    coltitle=white,      % cor do texto do título
    colbacktitle=black,  % fundo da barra de título preta
    sharp corners        % cantos retos
  }
}

%====== AMBIENTES DE TEOREMAS ESTILIZADOS ======%
\newtcbtheorem[number within=section]{teorema}{Teorema}{mytheostyle}{th}
\newtcbtheorem[number within=section]{corolario}{Corolário}{mytheostyle}{cor}
\newtcbtheorem[number within=section]{lema}{Lema}{mytheostyle}{lem}
\newtcbtheorem[number within=section]{proposicao}{Proposição}{mytheostyle}{prop}
\newtcbtheorem[number within=section]{definicao}{Definição}{mytheostyle}{def}
\newtcbtheorem[number within=section]{exemplo}{Exemplo}{mytheostyle}{ex}
\newtcbtheorem[number within=section]{observacao}{Observação}{mytheostyle}{obs}

% Observação:
% - Todos numeram por seção (ex.: Teorema 2.1)
% - “Exemplo” e “Observação” são úteis em notas de aula
% - Suporte a título opcional: 
%   \begin{definicao}[Título opcional] ... \end{definicao}
