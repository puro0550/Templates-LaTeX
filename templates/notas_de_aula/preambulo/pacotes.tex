%======================================
% PACOTES PRINCIPAIS
%======================================
\usepackage{graphicx}    % Inclusão de figuras
\usepackage[top=3cm,bottom=2cm,inner=3cm,outer=2cm,twoside]{geometry} % Margens
\usepackage{ragged2e}    % Melhor controle de alinhamento (ex.: \justifying)
\usepackage{fontspec}    % Necessário para XeLaTeX/LuaLaTeX (uso de fontes do sistema)
\usepackage{setspace}    % Espaçamento entre linhas (ex.: \onehalfspacing)
\usepackage{comment}     % Permite comentar blocos grandes de texto
\usepackage{fancyhdr}    % Personalização de cabeçalhos e rodapés
%\usepackage{lipsum}      % Geração de texto fictício (apenas para testes)
\usepackage[bottom]{footmisc} % Faz notas de rodapé ficarem sempre no fim da página
\usepackage{pdfpages}    % Inclusão de PDFs inteiros/páginas específicas
%\usepackage{csquotes}    % Citações melhor formatadas
\usepackage{booktabs}    % Tabelas mais elegantes
\usepackage{changepage}  % Ajustar margens localmente (útil para figuras largas)
\usepackage{bookmark}
\usepackage[most]{tcolorbox} % Para caixas estilizadas
\usepackage{enumitem} % Para enumerates customizados


%======================================
% PACOTES MATEMÁTICOS E GERAIS
%======================================
\usepackage{hyperref}    % Links clicáveis no PDF (índice, refs, URLs)
\usepackage{amssymb}     % Símbolos matemáticos
\usepackage{amsmath}     % Ambientes matemáticos (equation, align, etc.)
\usepackage{amsthm}      % Ambientes de teorema, lema, etc.
\usepackage{array}       % Colunas mais flexíveis em tabelas
\usepackage{unicode-math}% Uso de fontes matemáticas modernas (XeLaTeX/LuaLaTeX)
\usepackage{nameref}     % Referências pelo nome (ex.: \nameref{sec:intro})
\usepackage{float}       % Melhor controle de posicionamento de figuras/tabelas
\usepackage[table]{xcolor} % Cores em tabelas e no texto

%======================================
% OUTROS PACOTES
%======================================
\usepackage{soul}        % Destaque de texto (ex.: \hl{})
\usepackage{xcolor}      % Definição de cores
%\usepackage{tikz}        % Diagramas, esquemas e figuras vetoriais
%\usetikzlibrary{matrix,positioning,arrows.meta,fit,backgrounds} % Extensões do TikZ

% Carregar depois de hyperref e babel
\usepackage{newfloat}    % Criação de novos ambientes flutuantes
\usepackage{caption}     % Customização de legendas
\usepackage{tocloft}     % Customização do sumário/listas

%======================================
% NOTA SOBRE USO
%======================================
% Pacotes recomendados para "Notas de Aula":
% - graphicx, geometry, ragged2e, fontspec, setspace, fancyhdr
% - amsmath, amssymb, amsthm, xcolor, float, hyperref
% - tikz (se for usar esquemas)
%
% Pacotes que provavelmente NÃO serão usados em notas:
% - biblatex, etoolbox, csquotes (referências ABNT)
% - pdfpages (inserir PDFs inteiros)
% - lipsum (texto fictício, só teste)
% - tocloft (mais útil em TCCs/dissertações)
%
% Sugestão: manter o arquivo igual ao do TCC e só comentar
% o que não for necessário no momento.

